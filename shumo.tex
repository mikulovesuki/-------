\documentclass{article}
\usepackage{ctex}
\usepackage{mathtools}
\usepackage{amsthm}
\usepackage{amsfonts}
\usepackage{float}
\usepackage{tikz}
\usepackage{graphicx}
\usepackage{indentfirst} % 使首段也缩进
\usepackage{forest}
\usepackage{listings}
\usepackage{xcolor}
\usepackage{setspace} % 添加以支持 spacing 环境
\lstset{
    backgroundcolor=\color{gray!10}, % 背景色
    basicstyle=\ttfamily\small,      % 代码字体
    keywordstyle=\color{blue},       % 关键字颜色
    commentstyle=\color{gray},       % 注释颜色
    stringstyle=\color{red!70!black},% 字符串颜色
    numbers=left,                    % 行号显示在左侧
    numberstyle=\tiny\color{gray},
    breaklines=true,                 % 自动换行
    frame=single,                    % 框架
    tabsize=4,
    language=Python
}
\setlength{\parindent}{2em} % 设置首行缩进为两个汉字宽度
\everymath{\displaystyle}
\title{\large 基于动态规划的烟雾弹掩护目标规划模型(临时标题)}
\date{}
\begin{document}
\maketitle
\begin{center}
\section*{摘要}
\end{center}
\vspace{-1em}


\noindent


\noindent
\textbf{关键词:}\textit{关键词一},\textit{关键词二},\textit{关键词三},\textit{},\textit{}



\section{问题复述}

\subsection{问题背景}

在军事上,烟雾弹是一种用于掩护特定保护目标的战术装备。该装备通常搭配假干扰目标进行使用。\textit{烟幕干扰弹通过释放大量烟雾,形成视觉屏障,阻挡敌方的视线,以达到保护真实目标的目的。}在本材料中,烟幕干扰弹由无人机进行投掷。敌人发射多枚朝向假目标速度为$300m/s$的导弹,而我方的任务是通过投掷烟雾弹来掩护真实目标。在材料中,假目标和真实目标均为静止目标。假目标的坐标位置和真实目标的坐标位置均已知,其中假目标位于坐标轴原点,真目标位于$(0,200,0)$处,真目标为半径$7m$高度$10m$的圆柱体。我方携带烟幕弹无人机的部署位置均已知。敌人发射导弹的时间和位置均已知。烟雾弹投掷后的一段时间后会产生烟雾,烟雾会在20秒内持续存在,行程一个以烟幕弹为圆心半径为10$m$的球形烟幕团,并且烟幕团会以$3 m/s$的速度匀速下落,
持续时间为20$s$。为实现对真实目标的有效掩护,我方需要合理规划烟雾弹的投掷位置和时间,对无人机进行合理的调动,以确保在敌人导弹飞行过程中,真实目标尽可能处于于烟幕团的掩护范围内。其中无人机的飞行速度和方向一旦确定后,在飞行过程中均不可更改。

\textit{初始模型坐标位置可视化:}

\subsection{问题一}

已知无人机FY1位于$(17800,0,1800)$处,现在敌人从$(20000,0,2000)$处发射一枚速度为$300m/s$的导弹,朝向假目标飞行。现操纵无人机FY1以$120m/s$的速度朝假目标方向移动,并且在$1.5s$后投掷一枚烟幕弹,投掷后$3.6s$后起爆,本研究需要建立模型,计算出有效遮蔽时长。
\subsection{问题二}

\subsection{问题三}

\subsection{问题四}

\subsection{问题五}

\section{符号说明}

\begin{table}[H]
\centering
\caption{符号说明表}
\begin{tabular}{|c|l|}
\hline
符号 & 含义 \\
\hline

\hline
\end{tabular}
\end{table}

\section{问题分析}
本问题限制条件较多,涉及到多个物理量的计算,且各个物理量之间存在较强的关联性,其中存在多组函数关系。为方便后续问题的研究分析,现对有效遮蔽时长模型进行分析。

\begin{itemize}
    \item 简化模型:本研究对初始数据进行分析,发现无人机投掷烟雾弹位置应该均为离真假目标较远处,以无人机FY1的投掷位置为例,投掷位置距离假目标$17800m$,距离真目标$17600m$,真假目标距离较近,在真目标、假目标和无人机投掷连线的夹角小于$3^\circ$因此本研究在离真目标远距离投弹的情况下,真目标和假目标可以视为重合的点。
    \item 有效遮挡的几何模型:在本研究中,导弹的飞行轨迹始终为指向原点的直线,现在我们对有效遮挡时的几何情况进行探讨。发现在导弹飞行过程中,若真目标处于烟幕团的投影范围内,则认为真目标被有效遮挡。表现为:\textit{导弹与真目标的连线为一条穿过球形烟幕团的弦}。导弹轨迹与水平面的夹角为$\theta$,定义导弹轨迹与其在水平面上的投影所在的平面为轨迹面,烟幕团球心距轨迹面距离为$d$,则有效遮挡的条件为$d<10$,所得到的有效遮挡长度为:
    \begin{center}
        $\displaystyle \frac{2\sqrt{10^2-d^2}}{\cos\theta}$
    \end{center}
    \item 有效遮挡时长的计算:在本研究中,导弹的飞行速度为$300m/s$,烟幕团下落速度为$3m/s$,则有效遮挡时长为:
\end{itemize}
\subsection{问题一分析}

\subsection{问题二分析}

\subsection{问题三分析}

\subsection{问题四分析}

\subsection{问题五分析}

\section{模型假设}

\subsection{问题一假设}

\subsection{问题二假设}

\subsection{问题三假设}

\subsection{问题四假设}

\subsection{问题五假设}

\section{模型建立与求解}

\subsection{问题一模型建立与求解}

\subsection{问题二模型建立与求解}

\subsection{问题三模型建立与求解}

\subsection{问题四模型建立与求解}

\subsection{问题五模型建立与求解}

\section{模型优缺点及展望}

\subsection{模型优点分析}

\subsection{模型缺点分析}

\subsection{模型展望}




\section*{附录}
\vspace{-1em}


\noindent

\end{document}
