\documentclass{article}
\usepackage{ctex}
\usepackage{mathtools}
\usepackage{amsthm}
\usepackage{amsfonts}
\usepackage{float}
\usepackage{tikz}
\usepackage{graphicx}
\usepackage{indentfirst} % 使首段也缩进
\usepackage{forest}
\usepackage{listings}
\usepackage{xcolor}
\usepackage{setspace} % 添加以支持 spacing 环境
\lstset{
    backgroundcolor=\color{gray!10}, % 背景色
    basicstyle=\ttfamily\small,      % 代码字体
    keywordstyle=\color{blue},       % 关键字颜色
    commentstyle=\color{gray},       % 注释颜色
    stringstyle=\color{red!70!black},% 字符串颜色
    numbers=left,                    % 行号显示在左侧
    numberstyle=\tiny\color{gray},
    breaklines=true,                 % 自动换行
    frame=single,                    % 框架
    tabsize=4,
    language=Python
}
\setlength{\parindent}{2em} % 设置首行缩进为两个汉字宽度
\everymath{\displaystyle}
\title{\large 基于动态规划的烟雾弹掩护目标规划模型(临时标题)}
\date{}
\begin{document}
\maketitle
\begin{center}
\section*{摘要}
\end{center}
\vspace{-1em}


\noindent


\noindent
\textbf{关键词:}\textit{关键词一},\textit{关键词二},\textit{关键词三},\textit{},\textit{}



\section{问题复述}

\subsection{问题背景}

\textit{初始模型坐标位置分布图:}

\subsection{问题一}

\subsection{问题二}

\subsection{问题三}

\subsection{问题四}

\subsection{问题五}

\section{符号说明}

\begin{table}[H]
\centering
\caption{符号说明表}
\begin{tabular}{|c|l|}
\hline
符号 & 含义 \\
\hline

\hline
\end{tabular}
\end{table}

\section{问题分析}

\subsection{问题一分析}

\subsection{问题二分析}

\subsection{问题三分析}

\subsection{问题四分析}

\subsection{问题五分析}

\section{模型假设}

\subsection{问题一假设}

\subsection{问题二假设}

\subsection{问题三假设}

\subsection{问题四假设}

\subsection{问题五假设}

\section{模型建立与求解}

\subsection{问题一模型建立与求解}

\subsection{问题二模型建立与求解}

\subsection{问题三模型建立与求解}

\subsection{问题四模型建立与求解}

\subsection{问题五模型建立与求解}

\section{模型优缺点及展望}

\subsection{模型优点分析}

\subsection{模型缺点分析}

\subsection{模型展望}




\section*{附录}
\vspace{-1em}


\noindent

\end{document}
